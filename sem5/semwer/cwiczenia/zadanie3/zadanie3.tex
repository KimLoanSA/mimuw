\documentclass[final,12pt]{article}
\usepackage{a4wide}
\usepackage{amsmath}
\usepackage{amssymb}
\usepackage{latexsym}
\usepackage[polish]{babel}
\usepackage[T1]{fontenc}
\usepackage[utf8]{inputenc}
\usepackage[pdftex]{graphicx}
\usepackage[pdftex]{color}
\usepackage[final]{listings}

%%%%%%%%%%%%% listings
\lstdefinelanguage{whileprograms}{morekeywords={while,do,if,then,else,div),(,decr,in,wrt},%
   sensitive,%
   morecomment=[l]//,%
   morecomment=[s]{\{}{\}},%
%   morecomment=[s]{$}{$},%
   basicstyle=\Large\tt,
   keywordstyle=\normalfont\bfseries\color{black},
   commentstyle=\color{blue},
   mathescape=true,
}

\lstset{language=whileprograms,flexiblecolumns=true,mathescape=true,frame=none}
\lstset{commentstyle=\it,basicstyle=\tt}
\lstset{literate={<=}{{$\leq$}}1
      {>=}{{$\geq$}}1
      {^}{{$\land$}{$\;$}}2
      {&}{{$\land$}{$\;$}}2
      {||}{{$\lor$}{$\;$}}2
      {<>}{{$\neq$}{$\;$}}2
      {=>}{{$\Rightarrow$}{$\;$}}2
      {EE}{{$\exists$}}1
      {AA}{{$\forall$}}1
      {!}{{$\neg$}}1
}

\title{\vspace{-2cm}
Semantyka i Weryfikacja programów \\
\large Praca domowa 3.
}
\author{Marcin Abramowicz ma406058}


\begin{document}

   \maketitle

   \section{Zadanie}

   \lstset{language=whileprograms} \selectlanguage{polish}

   Rozważamy proste rozszerzenie języka {\sc Tiny} o jednowymiarowe
   tablice.

   Dany jest następujący program w tym języku, operujący na tablicy
   $A$ {\bf indeksowanej od $1$ do $n$}.
   Należy udowodnić częściową poprawność
   tego programu względem
   podanych warunków. Wymagane jest podanie niezmienników obu
   pętli programu oraz asercji pośrednich.

   Poniższa specyfikacja używa następującej formuły, stwierdzającej że fragment tablicy $A$ jest posortowany niemalejąco:
   \begin{eqnarray*}
      S(i,j) &\equiv&  \forall p. (i\leq p<j \Rightarrow A[p]\leq A[p+1])
   \end{eqnarray*}

   \begin{lstlisting}
            {n>0 & AAp.(1<=p<=n =>  A[p]>=0)}
            i:=1;
            while
              (i<n) do
                j:=i+1;
                while
                  (j<=n)  do
                   if A[i]>A[j] then
                       t:=A[i]; A[i]:=A[j]; A[j]:=t
                   else
                       j:=j+1;
                i:=i+1
            {S(1,n)}
   \end{lstlisting}


   \section{Rozwiązanie}

   \begin{lstlisting}
    {n>0 & AAp.(1<=p<=n => A[p]>=0)}
    i:=1;
    while {S(1,i) & AAp.((i>1 & i<=p<=n) => A[i-1]<=A[p])}
      (i<n) do
        {i<n & S(1,i) & AAp.((i>1 & i<=p<=n) => A[i-1]<=A[p])}
        j:=i+1;
        {i<n & j=i+1 & j<=n}
        while {AAp.(i<p<j => A[i]<=A[p])}
          (j<=n)  do
           {i<j<=n & AAp.(i<p<j => A[i]<=A[p])}
           if A[i]>A[j] then
               {i<j<=n & A[i]>A[j] & AAp.(i<p<j => A[i]<=A[p])}
               t:=A[i]; A[i]:=A[j]; A[j]:=t
           else
               {i<j<=n & AAp.(i<p<=j => A[i]<=A[p])}
               j:=j+1;
        {j=n+1 & S(1,i+1) & AAp.(i<p<=n => A[i]<=A[p])}
        i:=i+1
    {S(1,n)}
   \end{lstlisting}


\end{document}

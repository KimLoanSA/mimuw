% !TeX spellcheck = pl_PL
%!TEX encoding =  UTF-8 Unicode

\documentclass{article}
\usepackage{amssymb}
\usepackage{amsmath}

\usepackage[all]{xy}

\usepackage{graphicx}
\usepackage[bookmarks = true, pagebackref = false]{hyperref}
\usepackage[T1]{fontenc}
\usepackage[utf8]{inputenc}
\usepackage[polish]{babel}

\newcommand{\note}[1]{\marginpar{\raggedright\footnotesize\tt #1}}

\newcommand{\otto}{\leftrightarrow}
\newcommand{\sembr}[1]{[\![#1]\!]}

\usepackage{amssymb}
\usepackage{amsmath}
\usepackage[dvipsnames]{xcolor}
\usepackage{empheq}
\usepackage{bm}

\newcommand{\pto}{\rightharpoonup}

\newcommand{\syn}[1]{{\bf \textcolor{RawSienna}{#1}}}
\newcommand{\sem}[1]{{\bf \textcolor{olive}{#1}}}
\newcommand{\semcol}[1]{{\textcolor{olive}{#1}}}
\newcommand{\sembr}[1]{[\![#1]\!]}
\newcommand{\comm}[1]{\quad \; \text{ \textcolor{black!42}{ - #1} } }

\newcommand{\boxedeq}[1]{\begin{empheq}[box={\fboxsep=6pt\fbox}]{align*}#1\end{empheq}}

\newcommand{\bigeps}{\mathcal{E}}

\newcommand{\boxedeq}[1]{\begin{empheq}[box={\fboxsep=6pt\fbox}]{align*}#1\end{empheq}}

\newcommand{\bigeps}{\mathcal{E}}

\title{
   Semantyka i Weryfikacja programów \\
   \large Egzamin, zadanie 2.
}
\author{Marcin Abramowicz ma406058}

\begin{document}

   \maketitle

   \section{Zadanie}

   Napisz semantykę denotacyjną, w stylu kontynuacyjnym, instrukcji języka o gramatyce:
   \begin{alignat*}{4}
      Num &\ni n & & ::= &\ & {\tt 0} \mid{\tt 1} \mid {\tt -1} \mid {\tt 2} \mid {\tt -2} \mid \cdots \\
      Var &\ni x && ::= && {\tt x}\mid {\tt y}\mid \cdots \\
      PName &\ni p & & ::= && {\tt p}\mid {\tt q}\mid \cdots \\
      Expr &\ni e && ::= && n \mid x \mid e_1+e_2 \mid e_1*e_2 \mid e_1-e_2 \\
      BExpr &\ni b && ::= && {\tt true} \mid {\tt false} \mid e_1<e_2 \mid e_1=e_2 \mid b_1\land b_2 \mid {\tt not}\ b\\
      VDecl &\ni d_V && ::= && {\tt var}\ x \\
      PDecl &\ni d_P && ::= && {\tt proc}\ p\ (I) \\
      Instr &\ni I && ::= && x := e \mid I_1;I_2 \mid {\tt if}\ b\ {\tt then}\ I \mid {\tt while}\ b\ {\tt do}\ I \mid {\tt begin}\ d_V; d_P;  I\ {\tt end} \mid \\
      &&&&& {\tt call}\ p \mid {\tt break} \mid {\tt continue} \mid {\tt exit} %\\
      %Prog &\ni P && ::= && {\tt program}\ I
   \end{alignat*}

   Jest to język z rekurencyjnymi procedurami bezparametrowymi oraz z prostymi mechanizmami przerywania pętli i procedur.

   Instrukcja {\tt break} powoduje zakończenie wykonywania najbliższej aktualnie wykonywanej pętli i przejście do dalszego wykonywania programu. Dzieje się tak nawet wtedy, jeśli instrukcja {\tt break} nie występuje w ciele tej pętli, ale w procedurze wywołanej (bespośrednio lub poprzez inne procedury) z tej pętli. Na przykład, w programie:
   \begin{verbatim}
        begin
           var x;
           proc p (if x>3 then break);
           x := 1;
           while true do
             call(p);
             x := x+1
        end
   \end{verbatim}
   wykonanie pętli kończy się, a zmienna {\tt x} przyjmuje wartość $4$.

   Instrukcja {\tt continue} powoduje zakończenie wykonywania aktualnej iteracji najbliższej aktualnie wykonywanej pętli i przejście do wykonywania następnej iteracji tej pętli, z uwagami jak dla instrukcji {\tt break} powyżej. Na przykład, gdyby w powyższym przykładzie zastąpić instrukcję {\tt break} instrukcją {\tt continue}, to program zapętli się w stanie pamięci, w którym zmienna {\tt x} przyjmuje wartość $4$.

   Instrukcja {\tt exit} powoduje zakończenie wykonywania najbliższej aktualnie wykonywanej procedury. Gdyby w powyższym przykładzie zastąpić instrukcję {\tt break} instrukcją {\tt exit}, to program zapętli się, a zmienna {\tt x} będzie przyjmować coraz większe wartości.

   Semantyka pozostałych konstrukcji jest standardowa. Widoczność wszystkich identyfikatorów jest statyczna.

   W rozwiązaniu należy podać co najmniej równania semantyczne dla instrukcji {\tt break}, {\tt continue}, {\tt exit}, ${\tt while..do..}$ i {\tt call}, a także dla deklaracji procedur. Pozostałe równania można pominąć, jeżeli są standardowe lub zupełnie analogiczne do standardowych. Można założyć, że dana jest funkcja semantyczna dla wyrażeń i wyrażeń logicznych, w wybranym przez siebie wariancie -- klasycznym denotacyjnym lub w stylu kontynuacyjnym. Należy natomiast zdefiniować wszystkie potrzebne dziedziny pomocnicze i typy wszystkich funkcji semantycznych.

   Można założyć, że programy nie korzystają z niezadeklarowanych zmiennych ani procedur. Można też założyć, że instrukcja {\tt exit} może wystąpić tylko w ciele jakiejś procedury, a instrukcje {\tt break} i {\tt continue} będą wykonywane tylko podczas wykonywania jakiejś pętli.


   \newpage
   \section{Rozwiązanie}

   Będziemy modyfikować jedynie semantykę dla $Instr$ i $PDecl$, reszta pozostaje standardowa
   (jak na wykładzie).
   Rozwiązanie będzie się wzorować na zadaniu z {\bf continue} oraz {\bf break} omawiane na ćwiczeniach.

   \subsection{Dziedziny semantyczne}

   Wszystkie niepodane dziedziny są takie same jak na wykładzie.

   \begin{gather*}
      \begin{aligned}
         \sem{Cont}_{\sem{break}} &= \sem{Cont} \\
         \sem{Cont}_{\sem{exit}} &= \sem{Cont} \\
         \sem{Cont}_{\sem{continue}} &= \sem{Cont} \\
         \\
         \sem{PDECL} &=
         \sem{VEnv} \to
         \sem{PEnv} \to
         \sem{Cont} \\
         \sem{INSTR} &=
         \sem{VEnv} \to
         \sem{PEnv} \to
         \sem{Cont}_{\sem{break}} \to
         \sem{Cont}_{\sem{exit}} \to
         \sem{Cont}_{\sem{continue}} \to
         \sem{Cont} \\
      \end{aligned}
   \end{gather*}


   \subsection{Funkcje semantyczne}

   Wszystkie niepodane funkcje są takie same jak na wykładzie.

   \begin{gather*}
      \semcol{\mathcal{D}_{V}}: \syn{PDecl} \to \sem{PDECL} \\
      \semcol{\mathcal{I}}: \syn{Instr} \to \sem{INSTR} \\
   \end{gather*}


   \subsection{Równania semantyczne}

   Równania semantyczne dla pozostałych operacji są analogiczne jak na wykładzie.

   \boxedeq{ \semcol{\mathcal{D}_{V}}: \syn{PDecl} \to \sem{PDECL} }

   Oczywiście wywołanie metody {\bf P} powinno zostać zapisane przy użyciu punku stałego.

   \begin{gather*}
      \semcol{\mathcal{D}_{V}}\sembr{\syn{proc} \ p \ (I)}
      \ \rho_{V} \ \rho_{P} =
      \rho_{P}[p \mapsto P], \\
      {\bf gdzie:} \\
      P =
      \lambda \beta_{break}. \
      \lambda \beta_{exit}. \
      \lambda \beta_{continue}. \
      \lambda \kappa.
      \semcol{\mathcal{I}}\sembr{I}
      \ \rho_{V} \ \rho_{P}[p \mapsto P] \ \beta_{break} \ \beta_{exit} \ \beta_{continue} \ \kappa
   \end{gather*}

   \boxedeq{ \semcol{\mathcal{I}}: \syn{Instr} \to \sem{INSTR} }

   Oczywiście wywołanie rekurencyjne w {\bf while} powinno być zapisane przy użyciu punktu stałego:
   \begin{gather*}
      \semcol{\mathcal{I}}\sembr{\syn{while} \ b \ \syn{do} \ I}
      \ \rho_{V} \ \rho_{P} \ \beta_{break} \ \beta_{exit} \ \beta_{continue} \ \kappa \ s = \\
      \semcol{\mathcal{B}}\sembr{b} \ \rho_{V} \ (\lambda v.
      ifte(v,
      \semcol{\mathcal{I}}\sembr{I}
      \ \rho_{V} \ \rho_{P} \ \kappa \ \kappa \ \beta \ \beta \ s, \kappa \ s)) \ s, \\
      {\bf gdzie:} \\
      \beta = \semcol{\mathcal{I}}\sembr{\syn{while}\ b \ \syn{do} \ I}
      \ \rho_{V} \ \rho_{P} \ \beta_{break} \ \beta_{exit} \ \beta_{continue} \ \kappa
      \\
      \\
      \\
      \semcol{\mathcal{I}}\sembr{\syn{break}}
      \ \rho_{V} \ \rho_{P} \ \beta_{break} \ \beta_{exit} \ \beta_{continue} \ \kappa \ =
      \beta_{break}
      \\
      \\
      \\
      \semcol{\mathcal{I}}\sembr{\syn{exit}}
      \ \rho_{V} \ \rho_{P} \ \beta_{break} \ \beta_{exit} \ \beta_{continue} \ \kappa \ =
      \beta_{exit}
      \\
      \\
      \\
      \semcol{\mathcal{I}}\sembr{\syn{continue}}
      \ \rho_{V} \ \rho_{P} \ \beta_{break} \ \beta_{exit} \ \beta_{continue} \ \kappa \ =
      \beta_{continue}
      \\
      \\
      \\
      \semcol{\mathcal{I}}\sembr{\syn{call} \ p}
      \ \rho_{V} \ \rho_{P} \ \beta_{break} \ \beta_{exit} \ \beta_{continue} \ \kappa \ =
      (\rho_{P} \ p) \ \beta_{break} \ \kappa \ \beta_{continue} \ \kappa \
   \end{gather*}


\end{document}

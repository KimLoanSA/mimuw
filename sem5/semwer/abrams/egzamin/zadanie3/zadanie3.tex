%!TEX encoding =  UTF-8 Unicode

\documentclass[final,11pt]{article}
\usepackage{a4wide}
\usepackage{amsmath}
\usepackage{amssymb}
\usepackage{latexsym}
\usepackage[polish,english]{babel}
\selectlanguage{english}
\usepackage[T1]{fontenc}
\usepackage[utf8]{inputenc}

\usepackage{url}
\usepackage{xspace}
\usepackage{verbatim}
\usepackage{alltt}
\usepackage{wrapfig}
\usepackage[pdftex]{graphicx}
\usepackage[pdftex]{color}
\usepackage[final]{listings}


%%%%%%%%%%%%% listings
\lstdefinelanguage{whileprograms}{morekeywords={while,do,if,then,else,div),(,decr,in,wrt},%
   sensitive,%
   morecomment=[l]//,%
   morecomment=[s]{\{}{\}},%
%   morecomment=[s]{$}{$},%
   basicstyle=\Large\tt,
   keywordstyle=\normalfont\bfseries\color{black},
   commentstyle=\color{blue},
   mathescape=true,
}

\lstset{language=whileprograms,flexiblecolumns=true,mathescape=true,frame=none}
\lstset{commentstyle=\it,basicstyle=\tt}
\lstset{literate={<=}{{$\leq$}}1
      {>=}{{$\geq$}}1
      {^}{{$\land$}{$\;$}}2
      {&}{{$\land$}{$\;$}}2
      {||}{{$\lor$}{$\;$}}2
      {<>}{{$\neq$}{$\;$}}2
      {=>}{{$\Rightarrow$}{$\;$}}2
      {EE}{{$\exists$}}1
      {AA}{{$\forall$}}1
      {!}{{$\neg$}}1
}

\title{
   Semantyka i Weryfikacja programów \\
   \large Egzamin, zadanie 3.
}
\author{Marcin Abramowicz ma406058}

\begin{document}

   \lstset{language=whileprograms} \selectlanguage{polish}

   \maketitle

   \section{Zadanie}

   Rozważamy proste rozszerzenie języka {\sc Tiny} o jednowymiarowe
   tablice.

   Dany jest następujący program w tym języku, operujący na tablicy
   $A$ {\bf indeksowanej od $1$ do $n$}.
   Należy udowodnić częściową poprawność
   tego programu względem
   podanych warunków. Wymagane jest podanie asercji we wszystkich wskazanych miejscach, w tym niezmienników dwóch z trzech pętli. (Miejsca na wpisanie niezmienników znajdują się bezpośrednio po słowach kluczowych {\bf while}.) Jeżeli ktoś uzna za stosowne, to można także podać więcej asercji.

   Poniższa specyfikacja używa następującej formuły, stwierdzającej że fragment tablicy $A$ jest posortowany niemalejąco:

   \begin{eqnarray*}
      S(i,j) &\equiv&  \forall p. (i\leq p<j \Rightarrow A[p]\leq A[p+1])
   \end{eqnarray*}

   \newpage
   \section{Rozwiązanie}

   \begin{lstlisting}
        {1<=k<=n &  S(1,k-1) & S(k,n) (s)}
        i:=1;
        while {S(1,k-1) & S(k,n)
               & 1<=k<=n+1 & 1<=i<=k (1)}
          (k<=n) do
            {1<=i<=k & S(1,k-1) & S(k,n)
             & 1<=k<=n (2)}
            while (A[i]<A[k]) do  i:=i+1;
            {1<=i<=k & 1<=k<=n & S(1,k-1) & S(k,n)
             & A[i]>=A[k] (3)}
            j:=i;
            while {1<=i<=k & 1<=k<=n
                   & 1<=j<=k & S(1,j) & S(k+1,n) & A[j]>=A[k] (4)}
              (j<k)  do
               {1<=i<=k & 1<=k<=n & S(1,j) & S(k+1,n) & A[j]>=A[k]
                & 1<=j<k (5)}
               t:=A[j]; A[j]:=A[k]; A[k]:=t;
               {1<=i<=k & 1<=k<=n & 1<=j<k & S(k+1,n)
                & A[j]<=A[k] & S(1,j+1) (6)}
               j:=j+1;
            {1<=i<=k & S(k+1,n)
             & j=k & S(1,k) (7)}
            k:=k+1
        {S(1,n) (k)}
   \end{lstlisting}


   \subsection{Wyjaśnienie}

   W blokach {} pierwsza linia zawiera jedynie warunki ''przepisane'' z poprzedniego,
   natomiast nowe linie zawierają nowe warunki.

   \subsubsection{Bardzo ogólnie wynikanie}

   \begin{itemize}
      \item
      (s) => (1): dodaję jedynie warunki na $k$
      (pamiętając, że po pętli $k = n + 1$) i $i$

      \item
      (1) => (2): wzmacniam warunek na $k$ wiedząc, że spełniamy warunek pętli

      \item
      (2) => (3): wiem, że w poprzedzającym $while$ znalazłem takie $i$,
      że $A[i] >= A[k]$

      \item
      (3) => (4): dodaję odpowiednie warunki na $j$,
      dodatkowo zmniejszam siłę warunków na posorotowanie,
      żeby finalnie dla $j = k + 1$ również to zachodziło

      \item
      (4) => (5): wzmacniam warunek na $j$ wiedząc, że spełniamy warunek pętli

      \item
      (5) => (6): po wykonaniu swapa odwraca mi się nierówność między
      $A[j]$ i $A[k]$, dodatkowo ciąg o 1 dłużsy jest posortowany
      (po dodaniu 1 do $j$ otrzymuję niezmiennik)

      \item
      (6) => (7): wyszedłem z pętli, wiem że $j = k$, ciąg o 1 dłuższy jest posortowany i pasuje to do niezmiennika jeśli dodam 1 do $k$

      \item
      (7) => (k): dochodzę $k = n$, więc finalny warunek jest spełniony

   \end{itemize}


\end{document}

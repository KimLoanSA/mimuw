\documentclass[a4paper]{article}

\usepackage{amssymb,mathrsfs,amsmath,amscd,amsthm}
\usepackage[mathcal]{euscript}
\usepackage{stmaryrd}
\usepackage{graphics}
\usepackage{blindtext}
\usepackage{enumitem}
\usepackage{amsmath}
\usepackage{algorithm}
\usepackage[noend]{algpseudocode}
\usepackage{mathtools}
\usepackage{textcomp}
\usepackage{siunitx}
\usepackage{hyperref}

\usepackage[T1]{fontenc}
\usepackage[utf8]{inputenc}
\usepackage[polish]{babel}

\makeatletter
\makeatother

\RequirePackage{a4wide}

%%%%%%% makra do notacji

\renewcommand{\le}{\leqslant} % mniejsze bądź równe
\renewcommand{\ge}{\geqslant} % większe bądź równe
\renewcommand\qedsymbol{\scalebox{0.75}{$\blacksquare$}} % koniec dowodu
\renewcommand{\phi}{\varphi} % litera φ
\newcommand{\eps}{\varepsilon} % litera ε

\newcommand{\N}{\mathbb N} % liczby naturalne
\newcommand{\R}{\mathbb R} % liczby rzeczywiste
\newcommand{\I}{\mathbb I} % liczby rzeczywiste
\newcommand{\set}[1]{\{#1\}} % \set{1,2,3} to zbiór {1,2,3}

\renewcommand{\subset}{\subseteq} %symbol ⊆
\newcommand{\Longupdownarrow}{\Big\Updownarrow}

\title{\vspace{-1cm}
Metody Numeryczne \\
\large Praca domowa 4.
}
\author{
   Marcin Abramowicz ma406058 \\
   \small współpraca: jo406271
}

\begin{document}

   \maketitle

   \section*{Zadanie 1.}

   Niech $\Phi_{N+1}(x) = (x - x_{0}) \dots (x - x_{N})$, gdzie $a = x_{0} < x_{1} < \dots, x_{N} \le b$.
   Wykaż, że:
   \[
      \max_{x \in [a, b]} \left| \Phi_{N+1}(x) \right| \ge \frac{1}{2^{N}} \cdot \left( \frac{b - a}{2} \right)^{N+1},
   \]
   a równość jest osiągana na odpowiednio (jak?) przeskalowanych / przesuniętych węzłach Czebyszewa.
   Wywnioskuj stąd, że istnieją węzły $a \le x_{0} < x_{1} < \dots < x_{N} \le b$ takie,
   że błąd aproksymacji $f \in C^{N+1}[a, b]$ przez oparty na nich WIL $p$ spełnia
   \[
      \left\| f - p \right\|_{\infty} \le
      \frac{\left\| f^{(N+1)} \right\|_{\infty}}{2^{N} \cdot (N + 1)!} \cdot
      \left( \frac{b - a}{2} \right)^{N+1}.
   \]


   \section*{\large Rozwiązanie}

   Zacznijmy od udowodnienia ze dla węzłów Czebyszewa zachodzi równość.

   Z twierdzenia o minimaksie wiemy ze najmniejszą normę w przedziale $[-1, 1]$
   ma wielomian $\frac{T_{n}}{2^{n-1}} = \Phi^{*}_{n}$.
   Dodatkowo wiemy, że:
   \[
      \max_{x \in [-1, 1]} \left| T_{n}(x) \right| = 1
   \]
   Czyli dla minimalnego wielomianu:
   \[
      \max_{x \in [-1, 1]} \left| \frac{T_{n}(x)}{2^{n-1}} \right| = \frac{1}{2^{n-1}}
   \]
   Teraz jeśli przedział to $[a, b]$, a nie $[-1, 1]$ możemy ''zmapować'' go na $[-1, 1]$ i użyć wielomianu $\Phi^{*}$ otrzymanego z twierdzenia o minimaksie
   (''zmapowac'' każdy przedział na odpowiednią długość i przesunąć argumenty):
   \[
      \Phi_{N+1}(x) = \left( \frac{b - a}{2} \right)^{N + 1} \cdot \Phi^{*}_{N+1} \left( \frac{2x - a - b}{b - a} \right)
   \]
   Z tego wszystkiego otrzymujemy, że jeśli odpowiednio przeskalujemy węzły to otrzymamy równość:
   \[
      \max_{x \in [a, b]} \left| \Phi_{N+1}(x) \right| =
      \left( \frac{b - a}{2} \right)^{N + 1} \cdot \frac{1}{2^{N}}
   \]
   Ciągle rozpatrywaliśmy wielomian z najmniejszą normą, co oznacza, że każdy inny nie będący nim będzie miał ją większą.
   Czyli, wtedy nie będzie już równości a jedynie nierówność z zadania, co kończy tą część dowodu. \\
   \\

   Aby udowodnić, błąd aproksymacji posłużymy się podobnym rozumowaniem.

   Z twierdzenia ze strony 194. wykładu, dla WIL $p$ i $g \in C^{N+1}[-1, 1]$ opartym na węzłach Czebyszewa:
   \[
      \left\| g - p \right\|_{\infty} \le \frac{\left\| g^{(N+1)}\right\|_{\infty}}{2^{N} \cdot (N + 1)!}
   \]
   Ponownie możemy zastanowić się jak ''zmapować'' funkcję $f \in C^{N+1}[a, b]$.
   Znowu możemy użyć funkcji $f^{*} \in C^{N+1}[-1, 1]$ podobnie jak z naszym $\Phi$.
   Wtedy nasza norma będzie wynosiła:
   \[
      \left\| f \right\|_{\infty} = \left( \frac{b - a}{2} \right)^{N+1} \cdot \left\| f^{*} \right\|_{\infty}
   \]
   Z czego otrzymujemy naszą nierówność:
   \[
      \left\| f - p \right\|_{\infty} \le
      \frac{\left\| f^{(N+1)} \right\|_{\infty}}{2^{N} \cdot (N + 1)!} \cdot
      \left( \frac{b - a}{2} \right)^{N+1}
   \]

   \section*{Zadanie 2.}

   Funkcję $f(x) = e^{x}$ aproksymujemy najlepiej, jak to możliwe, za pomocą wielomianu stopnia co najwyżej 1.
   Podaj wzór na taki wielomian najlepszej aproksymacji w sensie:
   \begin{itemize}
      \item
      średniokwadratowym na $(-1, 1)$, to znaczy w normie
      $\left\| f \right\| = \sqrt{\int^{1}_{-1} f^{2}(x) \ dx}$

      \item
      dyskretnie średniokwadratowym w punktach $\pm 1$, a konkretnie w normie
      $\left\| f \right\| = \sqrt{f^{2}(-1) + f^{2}(+1)}$.
   \end{itemize}
   W każdym przypadku podaj oszacowanie błedu aproksymacji.


   \section*{\large Rozwiązanie}

   \begin{itemize}
      \item
      Szukamy $f^{*}$ stopnia co najwyżej 1, takiego że najlepiej aproksymuje $f$ w sensie średniokwadratowym, czyli minimalizuje
      $\left\| f - f^{*} \right\|$,więc możemy minimalizować $\left\| f - f^{*} \right\|^{2}$:
      \begin{gather*}
         f^{*} = ax + b \\
         \\
         \left\| f - f^{*} \right\|^{2} = \\
         \int^{1}_{-1} (f - f^{*})^{2} \ dx = \\
         \int^{1}_{-1} (e^{x} - ax - b)^{2} \ dx = \\
         \frac{2 a^2}{3} + \frac{2b}{e} - \frac{4a}{e} + 2 b^{2} - 2 e b + \frac{e^2}{2} - \frac{1}{2 e ^{2}}
      \end{gather*}
      Będziemy minimalizować to wyrażenie.
      Zauważmy, że możemy minimalizować niezależnie składniki z $a$ i $b$:
      \begin{gather*}
         \min \left\{ \frac{2 a^2}{3} - \frac{4a}{e} \right\} =
         - \frac{6}{e^2}, \ a = \frac{3}{e} \\
         \\
         \min \left\{ \frac{2b}{e} + 2 b^{2} - 2 e b \right\} =
         1 - \frac{1}{2 e^{2}} - \frac{e^2}{2}, \ b = \frac{e^{2} - 1}{2e}
      \end{gather*}
      Czyli nasz poszukiwany wielomian to:
      \[
         f^{*} = \frac{3}{e} \cdot x + \frac{e^{2} - 1}{2e}
      \]
      Błąd aproksymacji:
      \begin{gather*}
         \left\| f - f^{*} \right\| = \\
         \sqrt{\int^{1}_{-1} \left( f - f^{*} \right)^{2} \ dx} = \\
         \sqrt{\int^{1}_{-1} \left( e^{x} - \frac{3}{e} \cdot x - \frac{e^{2} - 1}{2e} \right)^{2} \ dx} = \\
         \sqrt{1 - \frac{7}{e^{2}}}
      \end{gather*}
      \item
      Szukamy $f^{*}$ stopnia co najwyżej 1, takiego że najlepiej aproksymuje $f$ w sensie dyskretnie średniokwadratowym,
      czyli minimalizuje $\left\| f - f^{*} \right\|$, więc również będziemy minimalizować $\left\| f - f^{*} \right\|^{2}$:
      \begin{gather*}
         f^{*}(x) = ax + b \\
         \\
         \left\| f - f^{*} \right\|^{2} = \\
         (f(-1) - f^{*}(-1))^{2} + (f(1) - f^{*}(1))^{2}
      \end{gather*}
      Spróbujmy więc znaleść takie $a$ i $b$, że $f(-1) - f^{*}(-1) = 0$ oraz $f(1) - f^{*}(1) = 0$:
      \begin{gather*}
         (f(-1) - f^{*}(-1))^{2} = 0 \\
         (e^{-1} + a - b)^{2} = 0 \\
         e^{-1} + a - b = 0 \\
         \\
         (f(1) - f^{*}(1))^{2} = 0 \\
         (e - a - b)^{2} = 0 \\
         e - a - b = 0 \\
         \Downarrow \\
         \begin{cases}
            e^{-1} + a - b = 0 \\
            e - a - b = 0
         \end{cases}
         \\
         \Downarrow \\
         \begin{cases}
            a = \frac{e^{2} + 1}{2 e} \\
            b = \frac{e^{2} - 1}{2 e}
         \end{cases}
      \end{gather*}
      Czyli nasz wielomian to:
      \[
         f^{*}(x) = \frac{e^{2} + 1}{2 e} \cdot x + \frac{e^{2} - 1}{2 e}
      \]
      A błąd aproksymacji:
      \begin{gather*}
         \left\| f - f^{*} \right\| = \\
         \sqrt{(f(-1) - f^{*}(-1))^{2} + (f(1) - f^{*}(1))^{2}} = 0
      \end{gather*}
   \end{itemize}


   \section*{Zadanie 3.}

   Jaka liczba $n$ węzłów (a) Czebyszewa (b) równoodległych w $[-1, 1]$ wystarczy, by błąd aproksymacji
   $f(x) = e^{x}$ przez wielomian interpolacyjny Lagrange'a dla $f$ oparty na tych węzłach był mniejszy niż $\eps > 0$?
   Podaj prosty do sprawdzenia warunek na $n$.


   \section*{\large Rozwiązanie}

   \begin{enumerate}[label=(\alph*)]
      \item
      Z twierdzenia ze strony 194. wykładu wiemy, że:
      \[
         \left\| f - p \right\|_{\infty} \le \frac{\left\| f^{(N)}\right\|_{\infty}}{2^{N-1} N!}
      \]
      Więc w naszym przypadku:
      \[
         \frac{\left\| f^{(N)}\right\|_{\infty}}{2^{N-1} N!} =
         \frac{\left\| e^{x} \right\|_{\infty}}{2^{N-1} N!} =
         \frac{e}{2^{N-1} N!} < \eps
      \]

      \item
      Natomiast ze strony 190. wiemy:
      \[
         \left\| f - p \right\|_{\infty} \le
         \frac{\left\| f^{(N)} \right\|}{N!} \cdot \left\| \Phi_{N} \right\|_{\infty}
      \]
      W naszym przypadku maksymalna odległość punktów to 2 więc szacujemy z góry:
      \[
         \left\| \Phi_{N} \right\|_{\infty} \le 2^{N}
      \]
      Wtedy:
      \[
         \frac{\left\| f^{(N)} \right\|}{N!} \cdot \left\| \Phi_{N} \right\|_{\infty} =
         \frac{\left\| e^{x} \right\|}{N!} \cdot \left\| \Phi_{N} \right\|_{\infty} =
         \frac{e}{N!} \cdot \left\| \Phi_{N} \right\|_{\infty} \le
         \frac{e}{N!} \cdot 2^{N} < \eps
      \]
   \end{enumerate}


   \section*{Zadanie 4.}

   Jacek i Wacek zaimplementowali w Ocave metodę Newtona, a następnie,
   niezależnie od siebie przeprowadzili testy numeryczne swoich implementacji
   na tym samym komputerze z procesorem Intel Core i7 3.4 GHz i pamięcią operacyjną 16 GB RAM.
   Oto wyniki, jakie uzyskali (z braku miejsca, podajemu je z ograniczoną dokładnością):

   \[
      \begin{tabular}{ | c || c c || c c | }
         \hline
         &
         \multicolumn{2}{ c || }{Jacek} &
         \multicolumn{2}{ c | }{Wacek} \\
         \hline \hline
         równanie &
         \multicolumn{2}{ c || }{$\sin^{2}(x) = 0$} &
         \multicolumn{2}{ c | }{$\cos(x + \frac{\pi}{2})$ = 0} \\
         \hline
         dokł. rozw. &
         \multicolumn{2}{ c || }{$x^{*} = \pi$} &
         \multicolumn{2}{ c | }{$x^{*} = \pi$} \\
         \hline \hline
         iteracja $i$ &
         przybl. rozw. $x_{i}$ & Błąd $|x_{i} - x^{*}|$ &
         przybl. rozw. $x_{i}$ & Błąd $|x_{i} - x^{*}|$ \\
         \hline
         0 & 2.50e+00 & 6.42e-01 & 2.50e+00 & 6.42e-01 \\
         1 & 2.87e+00 & 2.68e-01 & 3.25e+00 & 1.05e-01 \\
         2 & 3.01e+00 & 1.31e-01 & 3.12e+00 & 2.59e-02 \\
         3 & 3.08e+00 & 6.50e-02 & 3.15e+00 & 6.43e-03 \\
         4 & 3.11e+00 & 3.25e-02 & 3.14e+00 & 1.60e-03 \\
         5 & 3.13e+00 & 1.62e-02 & 3.14e+00 & 3.96e-04 \\
         6 & 3.13e+00 & 8.11e-03 & 3.14e+00 & 9.84e-05 \\
         7 & 3.14e+00 & 4.05e-03 & 3.14e+00 & 2.44e-05 \\
         \hline
      \end{tabular}
   \]
   Czy możesz stwierdzić, że któryś z nich ma błąd w implementacji?
   Wyjaśnij swoje podejrzenia.

   \section*{\large Rozwiązanie}

   \subsubsection*{Sprawdzmy pochodne w obliczanych punktach}
   Jacek: $\left( \sin^{2} \right)'(\pi) = 0$ \\
   Wacek: $\cos'(\pi + \frac{\pi}{2}) = 1$ \\
   \\
   Jacek ma zerową pochodną czyli metoda może jedynie zbiegać liniowo (216. i 223. strona wykładu).
   Wacek natomiast ma zerową pochodą, czyli ma zbiegać kwadratowo.

   \subsubsection*{Zbieżność liniowa błędu Jacka}

   \begin{gather*}
      \frac{2.68e-01}{6.42e-01} \approx 0.417 \\
      \frac{2.68e-01}{6.42e-01} \approx 0.417 \\
      \frac{1.31e-01}{2.68e-01} \approx 0.488 \\
      \frac{6.50e-02}{1.31e-01} \approx 0.496 \\
      \frac{3.25e-02}{6.50e-02} \approx 0.500 \\
      \frac{1.62e-02}{3.25e-02} \approx 0.498 \\
      \frac{8.11e-03}{1.62e-02} \approx 0.500 \\
      \frac{4.05e-03}{8.11e-03} \approx 0.499 \\
   \end{gather*}
   Błąd Jacka zbiega liniowo.


   \subsubsection*{Zbieżność kwadratowa błędu Wacka}

   \begin{gather*}
      \frac{1.05e-01}{(6.42e-01)^{2}} \approx 0.254 \\
      \frac{2.59e-02}{(1.05e-01)^{2}} \approx 2.349 \\
      \frac{6.43e-03}{(2.59e-02)^{2}} \approx 9.585 \\
      \frac{1.60e-03}{(6.43e-03)^{2}} \approx 38.698 \\
      \frac{3.96e-04}{(1.60e-03)^{2}} \approx 154.687 \\
      \frac{9.84e-05}{(3.96e-04)^{2}} \approx 627.486 \\
      \frac{2.44e-05}{(9.84e-05)^{2}} \approx 2519.994 \\
   \end{gather*}
   Błąd Wacka jak widać nie zbiega kwadratowo.


   \subsubsection*{Wnioski}

   Jacek ma poprawną implementację - pochodna niezerowa,
   czyli chcemy żeby błąd zbiegał liniowo i tak się dzieje.

   Wacek natomiast ma błąd w swojej implementacji.
   Pochodna jest zerowa w punkcie, więc oczekujemy kwadratowej zbieżności błędu,
   ale jednak ona nie występuje.


\end{document}

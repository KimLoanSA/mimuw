\documentclass{article}
\usepackage[T1]{fontenc}
\usepackage[utf8]{inputenc}
\usepackage[polish]{babel}


\title{MIMUW WBO \\ Raport końcowy zadania zaliczeniowego nr 2.}
\author{Marcin Abramowicz ma406058}


\begin{document}

    \maketitle

    \section{Opis implementacji}

        \subsection{Wyszukiwanie}

            Sekwencje są:
            \begin{itemize}
                \item
                    parsowane przy pomocy `SeqIO.parse()`

                \item
                    tłumaczone przy pomocy `.translate()`

                \item
                    zapisywane do pliku przy pomocy `SeqIO.write()`

            \end{itemize}

            Następnie BLAST (są to bashowe komendy wykonywane z pythona):
            \begin{itemize}
                \item
                    jest tworzony przy pomocy `makeblastdb`

                \item
                    odpytywany przy pomocy `blastp`, użyłem dosyć arbitralnej e-wartości

                \item
                    zapisywane do pliku przy pomocy `SeqIO.write()`
            \end{itemize}

            Następnie, wyniki z pliku XML są parsowane przy pomocy `NCBIXML.parse()` i same sekwencje są ponownie zapisywane do pliku jak wyżej.


        \subsection{PFAM}

            HMMER (ponownie bash z pythona):
            \begin{itemize}
                \item
                    jest indeksowany przy pomocy `hmmpress`

                \item
                    skanowany przy pomocy `hmmscan`

            \end{itemize}

            Następnie wyniki są parsowane przy pomocy `SearchIO.parse()` i wypisywane są znalezione domeny.


        \subsection{Gene Ontology}

            \begin{itemize}
                \item
                    plik `.gaf` jest parsowany przy pomocy `gafiterator` z `Bio.UniProt.GOA`

                \item
                    GO natomiast jest parsowane przy pomocy `obo\_parser` z `goatools`

                \item
                    i dla wszystkich GO id, ogólniejsze terminy są znajdowane przy pomocy metody `paths\_to\_top`

            \end{itemize}


        \subsection{Test Fisher'a}

            Z wielkim smutkiem, ale brak.


    \section{Opis wyników}

        \subsection{Wyszukiwanie}

            Blast znalazł pełne sekwencje białkowe, czyli podane fragmenty dało się uliniowić z całymi sekwencjami E. Coli.
            \\ \\
            Przykład z pliku wynikowego:
    \begin{verbatim}
    >CCQ26909
        cdna chromosome:HUSEC2011CHR1:Chromosome:31576:32724:1
        gene:HUS2011_0030 gene_biotype:protein_coding
        transcript_biotype:protein_coding gene_symbol:carA
        description:carbamoyl phosphate synthetase small subunit,
        glutamine amidotransferase
    LIKSALLVLEDGTQFHGRAIGATGSAVGEVVFNTSMTGYQEILTDPSYSRQIVTLTYPHI
    GNVGTNDADEESSQVHAQGLVIRDLPLIASNFRNTEDLSSYLKRHNIVAIADIDTRKLTR
    LLREKGAQNGCIIAGDNPDAALALEKARAFPGLNGMDLAKEVTTAEAYSWTQGSWTLTGG
    LPEAKKEDELPFHVVAYDFGAKRNILRMLVDRGCRLTIVPAQTSAEDVLKMNPDGIFLSN
    GPGDPAPCDYAITAIQKFLETDIPVFGICLGHQLLALASGAKTVKMKFGHHGGNHPVKDV
    EKNVVMITAQNHGFAVDEATLPANLRVTHKSLFDGTLQGIHRTDKPAFSFQGHPEASPGP
    HDAAPLFDHFIELIEQYRKTAK*

    ...
    \end{verbatim}


        \subsection{PFAM}

            Sekwencje zawieraja nawet sporo domen, między innymi:
            \begin{verbatim}
                CPSase_sm_chain
                GATase
                Peptidase_C26
            \end{verbatim}


        \subsection{Gene Ontology}

            To co mnie zdziwiło, to że liczba uogólnień dla niektórych terminów jest naprawde spora, np dla `GO:0009264`


        \subsection{Test Fisher'a}

            Z wielkim smutkiem, ale tutaj również brak.


    \section{Podsumowanie}

        Projekt bardzo ciekawy, dający możliwość zetknięcia się z narzędziami i zobaczyć jak mogą działać wspólnie.

        Testu Fishera nie zaimplementowałem, z powodu odrobiny braku czasu z powodu innych projektów i licencjatu oraz odrobiny braku opanowania tego w pełni. Chciałbym w pełni zrozumieć co się w tych testach dokładnie dzieje, żeby móc je zaimplementować z pełnym zrozumieniem, ale niestety jak już wspomniałem, brak czasu.

        Dodatkowo, co widać po lakoniczności wypowiedzi w tym raporcie, muszę zdecydowanie popracować nad spisywaniem raportów. Zauważyłem, że zbieranie wyników i przedstawianie ich, nie wychodzi mi najlepiej, co zdecydowanie chciałbym poprawić.

        Na koniec, chciałbym dodać, że, okazja ''pobawienia'' się bliżej z sekwencjami, o których sie tylko czyta w książkach, jest bardzo ciekawe i zachęcające do dalszej przygody.


\end{document}

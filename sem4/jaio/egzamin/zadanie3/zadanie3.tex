\documentclass[a4paper]{article}

\usepackage{amssymb,mathrsfs,amsmath,amscd,amsthm}
\usepackage[mathcal]{euscript}
\usepackage{stmaryrd}
\usepackage[T1]{fontenc}
\usepackage[utf8]{inputenc}
\usepackage[polish]{babel}
\usepackage{graphics}

%%%%%%% makra do notacji

\renewcommand{\le}{\leqslant}%mniejsze bądź równe
\renewcommand{\ge}{\geqslant}%większe bądź równe
\renewcommand\qedsymbol{\scalebox{0.75}{$\blacksquare$}}%koniec dowodu
\newcommand\exendsymbol{\scalebox{1}{$\lrcorner$}}%inny koniec dowodu
\renewcommand{\phi}{\varphi}%litera φ
\newcommand{\eps}{\varepsilon}%litera ε
\newcommand{\nr}[1]{\smallcaps{NR#1}}%zadanie ze zbioru Niwiński-Rytter

\newcommand{\bin}{\textrm{bin}}%napis "bin" oznaczający binarną reprezentację liczby

\newcommand{\rev}{{\mathsf R}}%L^\rev to odwrócenie języka L
\newcommand{\N}{\mathbb N}%liczby naturalne
\newcommand{\set}[1]{\{#1\}}%\set{1,2,3} to zbiór {1,2,3}
\newcommand{\setof}[2]{\{#1\mid #2\}}%\setof{(x,y)}{x,y\in\N,x+y=5} to {(x,y)|x,y∈N, x+y=5}
\newcommand{\from}{\colon}%f\from X\to Y to funkcja f:X→Y

\renewcommand{\subset}{\subseteq}%symbol ⊆
\newcommand{\aut}[1]{\mathcal {#1}}%\aut A to automat A
\newcommand{\reg}[1]{\mathcal {#1}}%\reg E to wyrażenie regularne E
\newcommand{\gram}[1]{\mathcal {#1}}%\gram G to gramatyka G
\newcommand{\lang}{L}%\lang(\aut A) to L(A)

\newcommand{\tran}[1]{\xrightarrow{#1}}%p\tran a q to tranzycja z p do q z etykietą a

\newcommand{\produce}{\rightarrow}%produkcja np. X\produce YZ
\newcommand{\sep}{\mathop{\big|}}%X\produce YZ\sep ZT


%notacja dla lematu o pompowaniu dla CFL: długie słowo jest postaci \prefix \pleft \infix \pright \suffix
\newcommand{\prefix}{\mathit{prefix}}
\newcommand{\infix}{\mathit{infix}}
\newcommand{\suffix}{\mathit{suffix}}
\newcommand{\pleft}{\mathit{left}}
\newcommand{\pright}{\mathit{right}}

%tranzycje automatu ze stosem
\newcommand{\tranp}[3]{\xrightarrow{\textbf{pop}(#1), #2 ,\textbf{push}(#3)}}
\newcommand{\trant}[5]{#1,\textbf{read}(#2):\textbf{write}(#4),\textbf{state}(#3),\textbf{move}(#5)}

\newtheorem{twierdzenie}{Twierdzenie}
\newtheorem{fakt}{Fakt}
\newtheorem{wniosek}{Wniosek}
\newtheorem{lemat}{Lemat}
\newtheorem{zadanie}{Zadanie}
\newtheorem{zadanie*}{Zadanie$^*$}


\title{JAO Egzamin\\ Zadanie 3.}
\author{Marcin Abramowicz}

\begin{document}
\maketitle
Czy następujący problem jest rozstrzygalny?
\begin{description}
\item [Dane:]gramatyka bezkontekstowa $\gram G$ oraz słowa $u, v$ nad alfabetem terminali $\gram G$.
\item [Rozstrzygnąć:]czy $u\sim v$, gdzie $\sim$ to relacja Myhilla-Nerodego języka $\lang(\gram G)$?
\end{description}

\section*{Rozwiązanie}

Problem ten nie jest roztrzygalny, jest on nierozstrzygalny.

Udowodnimy to dzięki redukcji problemu uniwersalności gramatyki do naszego problemu. Załóżmy, przez sprzeczność, że nasz problem jest rozstrzygalny i spróbujemy go zredukować do problemu uniwersalności gramatyki.

Problem nasz jest rozstrzygalny i mamy ``magiczne pudełko'' z nim, które portafi odpowiadać na nasze zapytania. Spróbujemy rozstrzygnąć problem uniwersalności gramatyki $\gram H$ z jego pomocą. Najpierw sprawdźmy czy $\eps$ należy do naszej gramatyki $\gram H$. Odpytujemy się naszego pudełka i się dowiadujemy, czy należy, jeśli należy to sprawdzamy dalej, jeśli nie należy to znaczy, że to nie jest gramatyka uniwersalna. Zauważmy, że dzięki relacji Myhilla-Nerodego wiemy, że jeśli $\eps \sim \alpha$, dla $\alpha \in A$, gdzie $A$ to alfabet terminali $\gram G$, wtedy dla każdego słowa $w$ zachodzi $\eps w \in \lang \iff \alpha w \in \lang$, czyli słowo $w \in \lang$. Czyli  pytając się pudełka czy każda litera z alfabetu $A$ jest w relacji z $\eps$ sprawdzamy równocześnie, czy gramatyka jest uniwersalna, gdyż wykonując te zapytania dowiadujemy się czy wszystkie słowa należą do języka.

Wiemy ze problem uniwersalności gramatyki jest nierozstrzygalny, więc otrzymaliśmy sprawczność, więc problem uniwersalności gramatyki redukuje się do naszego, tym samym udowadniając ze nasz problem jest nierozstrzygalny.


\end{document}

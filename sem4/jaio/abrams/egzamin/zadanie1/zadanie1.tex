\documentclass[a4paper]{article}

\usepackage{amssymb,mathrsfs,amsmath,amscd,amsthm}
\usepackage[mathcal]{euscript}
\usepackage{stmaryrd}
\usepackage[T1]{fontenc}
\usepackage[utf8]{inputenc}
\usepackage[polish]{babel}
\usepackage{graphics}

%%%%%%% makra do notacji

\renewcommand{\le}{\leqslant}%mniejsze bądź równe
\renewcommand{\ge}{\geqslant}%większe bądź równe
\renewcommand\qedsymbol{\scalebox{0.75}{$\blacksquare$}}%koniec dowodu
\newcommand\exendsymbol{\scalebox{1}{$\lrcorner$}}%inny koniec dowodu
\renewcommand{\phi}{\varphi}%litera φ
\newcommand{\eps}{\varepsilon}%litera ε
\newcommand{\nr}[1]{\smallcaps{NR#1}}%zadanie ze zbioru Niwiński-Rytter

\newcommand{\bin}{\textrm{bin}}%napis "bin" oznaczający binarną reprezentację liczby

\newcommand{\rev}{{\mathsf R}}%L^\rev to odwrócenie języka L
\newcommand{\N}{\mathbb N}%liczby naturalne
\newcommand{\set}[1]{\{#1\}}%\set{1,2,3} to zbiór {1,2,3}
\newcommand{\setof}[2]{\{#1\mid #2\}}%\setof{(x,y)}{x,y\in\N,x+y=5} to {(x,y)|x,y∈N, x+y=5}
\newcommand{\from}{\colon}%f\from X\to Y to funkcja f:X→Y

\renewcommand{\subset}{\subseteq}%symbol ⊆
\newcommand{\aut}[1]{\mathcal {#1}}%\aut A to automat A
\newcommand{\reg}[1]{\mathcal {#1}}%\reg E to wyrażenie regularne E
\newcommand{\gram}[1]{\mathcal {#1}}%\gram G to gramatyka G
\newcommand{\lang}{L}%\lang(\aut A) to L(A)

\newcommand{\tran}[1]{\xrightarrow{#1}}%p\tran a q to tranzycja z p do q z etykietą a

\newcommand{\produce}{\rightarrow}%produkcja np. X\produce YZ
\newcommand{\sep}{\mathop{\big|}}%X\produce YZ\sep ZT


%notacja dla lematu o pompowaniu dla CFL: długie słowo jest postaci \prefix \pleft \infix \pright \suffix
\newcommand{\prefix}{\mathit{prefix}}
\newcommand{\infix}{\mathit{infix}}
\newcommand{\suffix}{\mathit{suffix}}
\newcommand{\pleft}{\mathit{left}}
\newcommand{\pright}{\mathit{right}}

%tranzycje automatu ze stosem
\newcommand{\tranp}[3]{\xrightarrow{\textbf{pop}(#1), #2 ,\textbf{push}(#3)}}
\newcommand{\trant}[5]{#1,\textbf{read}(#2):\textbf{write}(#4),\textbf{state}(#3),\textbf{move}(#5)}

\newtheorem{twierdzenie}{Twierdzenie}
\newtheorem{fakt}{Fakt}
\newtheorem{wniosek}{Wniosek}
\newtheorem{lemat}{Lemat}
\newtheorem{zadanie}{Zadanie}
\newtheorem{zadanie*}{Zadanie$^*$}


\title{JAO Egzamin\\ Zadanie 1.}
\author{Marcin Abramowicz}

\begin{document}
\maketitle

Dla słowa $w\in \set{a,b}^*$ definiujemy liczby:
\begin{align*}
    w_+&=\max \set{\#_av-\#_bv\mid v \text{ jest prefiksem }w}\\
    w_-&=\max \set{\#_bv-\#_av\mid v \text{ jest prefiksem }w}\\
    \delta w&=w_++w_-
\end{align*}
Przykładowo, dla $w=aaabbbbab$
mamy $w_+=3, w_-=1, \delta w=4$.
Ile klas abstrakcji ma relacja Myhilla-Nerodego języka wszystkich słów $w\in\set{a,b}^*$ takich, że $\delta w\leq 1000$?

\medskip\noindent
\emph{Uwaga:} Słowo $a_1\ldots a_n$ ma prefiksy $\eps,\quad a_1,\quad a_1a_2,\quad\ldots\quad a_1\cdots a_n$.


\section*{Rozwiązanie}
Niech opisem naszych klas abstrakcji, dla słowa $w \in \set{a, b}^*$ będzie para $<r_+(w), r_-(w)>$ (będziemy skrótowo oznaczać ją jako $r(w)$), gdzie 
\begin{align*}
    p_+(w)&=\#_aw-\#_bw \\
    p_-(w)&=\#_bw-\#_aw \\
\end{align*} 
oraz
\begin{align*}
   r_+(w)= w_+-p_+(w) \\
   r_-(w)=w_--p_-(w) \\
   r_+(w) + r_-(w) \leq 1000 \\
\end{align*} 

Każde słowo ma dokładnie jeden taki opis, gdyż wartości te dla całego słowa są ustalone, więc każde słowo ma przypasowną dokładnie jedną klasę abstrakcji.

Są do klasy abstrakcji, gdyż mając słowa $u, v \in \set{a,b}^*$, gdzie $r(u) = r(v)$, wtedy mając dowolne słowo $c \in \set{a,b}^*$ i dopisując je na koniec $u$ i $v$ otrzymujamy słowa $uc$ i $vc$, jeśli któreś z nich należy do naszego języka, wtedy drugie również należy, gdyż klasa opisuje słowa o tym samym ``bilansie'' liter $a$ i $b$ (ile musimy dołożyć jakiejś litery, żeby wyskoczyć za język), więc jeśli dodamy na koniec słowo, zmodyfikuje ono bilans słów $u$ i $v$ w ten sam sposób.

Dwie różne pary $r_1(u)$ i $r_2(v)$, gdzie $r_1(u) \ne r_2(v)$ i $u, v \in \set{a,b}^*$ opisują słowa, które nie są równoważne, gdyż dla słowa $u$, dla którego $r_1(u) = <0, 1000>$ i dla słowa $v$, dla którego $r_2(u) = <1000, 0>$ dodamy na koniec literę $b$. Wtedy $r_1(ub) = <1, 999>$ i $r_2(vb) = <1001, 0>$, czyli $ub$ należy do języka, natomiast $vb$ nie należy, czyli są to różne klasy abstrakcji.

Policzmy teraz ile jest klas. Możemy dla każdej klasy ustalić w jej opisie pierwszą wartość w parze, a następnie zsumować ile par "pasuje" do niej (czyli wartości w parach sie sumują do maksymalnie do 1000). Dla przykładu do pary $<998, 0>$ "pasują" pary $<998, 1>$ i $<998, 2>$. Prowadzi nas to do wzoru
$$\sum_{n=0}^{n\leq1000}\sum_{m=n}^{m\leq 1000} 1 = \sum_{n=0}^{n \leq 1000} n + 1 = \sum_{n=1}^{n \leq 1001} n = \frac{1001 * (1002)}{2} = 501501$$

\end{document}
